\documentclass{article}

\usepackage[english]{babel}

\usepackage[letterpaper,top=2cm,bottom=2cm,left=3cm,right=3cm,marginparwidth=1.75cm]{geometry}

\usepackage{amsmath}
\usepackage{graphicx}
\usepackage{tikz}
\usetikzlibrary{arrows}
\usetikzlibrary{arrows.meta,bending,chains}
\newcommand{\diff}{\mathop{}\!\mathrm{d}}
\usepackage[colorlinks=true, allcolors=blue]{hyperref}
\usepackage{imakeidx}
\makeindex[columns=3, title=Alphabetical Index, intoc]

\title{Fasi di un Progetto2}
\author{Lorenzo Sanseverino 5DSA}
\begin{document}

\section{Introduzione}
Un progetto nasce da una idea/opportunità per arrivare ad un risultato, ponendosi degli obiettivi.
Ogni obiettivo è raggiungibile mediante sforzi coordinati da parte del gruppo di lavoro seguendo delle fasi.
Generalmente le fasi si dividono in 4:

\begin{enumerate}
\item Concezione
\item Definizione
\item Realizzazione
\item Chiusura	
\end{enumerate}

\section{Concezione, analisi della fattibilità del progetto e tecniche di analisi}
La concezione di un progetto è la nascita dell'idea e la comprensione della sua fattibilità.
La concezione può essere divisa in sottoparti:

\subsection{Analisi Situazione Attuale}
Si descrive il contesto(\textcolor{blue}{dominio del software}) applicativo del progetto descrivendo le esigenze degli utenti sia interni che esterni.
Viene in oltre svolta l'\textbf{Analisi Situazione Attuale del Software}, vengono \textbf{Identificare i Vincoli di Origine Ambientale} quindi bisogna tenere conto di tutti i vincoli ambientali, normativi, temporali ed economici e chiedere consigli ad esperti del mestiere(notai, ambientalisti,investiri).
Si svolge una \textbf{Analisi della Realtà} in cui un vincolo diviene una opportunità, ossia si crea una situazione nuova e di successo.

Infine si \textbf{Definiscono gli obiettivi del progetto in termini quantitativi sintetici}:
Per definire gli obiettivi in maniera realistica e concreta, quindi senza andare in contro a perdite di tempo od uscire dai vincoli(vedi Fig \ref{t}) predisposti, ci sono delle tecniche come
la \textbf{\textcolor{blue}{S.M.A.R.T.}}:
\begin{itemize}
	\item \textbf{Specifici:} Deve essere dettagliato ed espresso chiaramente
	\item \textbf{Misurabili:} Quantificatore che indica la qualità (es. 10\% più grande...).
	\item \textbf{Accordati:} Deve essere concordato con tutti i membri del progetto.
	\item \textbf{Realistici:} Deve rispettare i vincoli del progetto (vedi Fig \ref{t}) e le 										capacità dei membri.
	\item \textbf{Temporalmente Definito:} Inserire una data di scandeza/consegna da rispettare.
\end{itemize}
Altra tecnica è la \textbf{\textcolor{blue}{S.W.O.T.}}, ossia Strenght, Weakness, Opportunities e Threat.
Questi punti costruscono una tabella che conterrà i vari punti di forza e debolezza sia dovuti a fattori interni(Strenght e Weakness) sia ad esterni(Opportunities e Threat):

\begin{center}
\begin{tabular}{ |c|c|c|c| } 
 \hline
 S & W & O & T \\ [0.5ex] 
 \hline\hline
 cell1 & cell2 & cell3 & cell4 \\
 cell5 & cell6 & cell7 & cell8 \\
 \hline
\end{tabular}
\end{center}



\subsection{Definizione di Massima del progetto:}
Essa si dividi in
\begin{itemize}
    \item \textbf{Definizione dei requisiti:} Sono le condizioni che il sistema deve rispettare. 		Compito del progettista elaboare una proposta di soluzione con lo scopo di
    \begin{enumerate}
        \item Identificare come deve fare il sistema informativo per rispondere alle esigenze della gente(\textbf{Requisiti funzionali}).
        \item Precisare i confini dell'applicazione e le modalità di iterazione con l'ambiente(\textbf{Requisiti di interfaccia}).
        \item .
        \item Definire l'elenco dei requisiti
        \item Elaborare le contradizione tra requisiti
        \item Identificare e mantenere il tracciamento tra requisiti utente e requisiti software.
    \end{enumerate}
    \item \textbf{Definire in linea di massima le specifiche del sistema:} architettura dei dati, architettura applicativa ed interfaccia utente.
    \item \textbf{Scelta delle modalità di realizzazione del progetto:} MakeOrBuy, riuso dei componenti esistenti, manutenzioni del sistema,formazione ed assistenza utente.
\end{itemize}

\end{document}