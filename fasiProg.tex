\documentclass{report}

\usepackage[english]{babel}

\usepackage[letterpaper,top=2cm,bottom=2cm,left=3cm,right=3cm,marginparwidth=1.75cm]{geometry}

\usepackage{amsmath}
\usepackage{graphicx}
\usepackage{tikz}
\usetikzlibrary{arrows}
\usetikzlibrary{arrows.meta,bending,chains}
\newcommand{\diff}{\mathop{}\!\mathrm{d}}
\usepackage[colorlinks=true, allcolors=blue]{hyperref}
\usepackage{imakeidx}
\usepackage{pgfplots}
\usepackage{hyperref}
\usepackage{bookmark}
\usepackage{booktabs}
\usepackage{pgfgantt}
\usepackage{multirow}
\usepackage{lscape}
\bookmarksetup{
  numbered,
  open
}
\pgfplotsset{width=10cm,compat=1.9}
\newcommand{\quantities}[1]{%
  \begin{tabular}{@{}c@{}}\strut#1\strut\end{tabular}%
}



\makeindex[columns=3, title=Alphabetical Index, intoc]

\title{Progetto e PMBOK}
\author{Lorenzo Sanseverino 5DSA}
\renewcommand*{\thesection}{\arabic{section}}


\begin{document}
\tableofcontents
\maketitle

\section{Introduzione alle Fasi di un Progetto}
Un progetto nasce da una idea/opportunità per arrivare ad un risultato, ponendosi degli obiettivi.
Ogni obiettivo è raggiungibile mediante sforzi coordinati da parte del gruppo di lavoro seguendo delle fasi.
Generalmente le fasi si dividono in 4:

\begin{enumerate}
\item Concezione
\item Definizione
\item Realizzazione
\item Chiusura	
\end{enumerate}

Ciò avviene per ottimizzare al massimo la produttività ed limitare al massimo gli sprechi.
Vi sono nelle varie fasi(sopratutto nella prima) un ampio studio di tutti i vincoli e delle opportunità.
Il vincolo consiste in qualcosa che rende difficile la realizzazione del progetto.
Una oppurtunità è soluzione non prevista nel progetto che integrata possa aumentare la qualità dell'ultimo.
Volendo le varie fasi possono essere viste sottoforma di grafico cartesiano:
sulla ascissa troviamo il tempo \(t\)(mesi/anni, dipende dalla scadenza) e sulle ordinate il costo \(c\)(rappresentabile come un ritardo ed uno spreco di soldi).
Il grafico risultante è il seguente:
\begin{figure}[h!]
\begin{tikzpicture}
\begin{axis}[
    axis lines = left,
    xlabel = \(t\),
    ylabel = {\(c(t)\)},
    xmin=0, xmax=12,
    ymin=0, ymax=100000,
]
\addplot [
    color=red,
]
coordinates {
    (0,0)(2,0)(4,500)(6,1000)(8,3000)(10,10000)(12,100000)
    };
\addlegendentry{Bad Year}
\\
\addplot [
    color=blue,
]
coordinates {
    (0,0)(2,0)(4,500)(6,1000)(8,1500)(10,1700)(12,4000)
    };
\addlegendentry{Good Year}

\end{axis}
\end{tikzpicture}
\caption{Grafico andamento del Progetto}
\label{grap1}
\end{figure}

Come si può vedere la grafico, la linea rossa ha avuto grossi ritardi con un enorme spreco in termini di soldi: ciò è dovuto ad una brutta organizzazione iniziale.
Invece la linea blu è una sorta di condizione ideale nel caso in cui tutto fosse perfetto(avere dei ritardi e delle perdite durante la realizzazione del progetto è quasi normale, se ovviamente non eccessive)

\section{Concezione, analisi della fattibilità del progetto e tecniche di analisi}
La concezione di un progetto è la nascita dell'idea e la comprensione della sua fattibilità.
La concezione può essere divisa in sottoparti:

\subsection{Analisi Situazione Attuale}
Si descrive il contesto(\textcolor{blue}{dominio del software}) applicativo del progetto descrivendo le esigenze degli utenti sia interni che esterni.\\
Vengono \textbf{Identificare i Vincoli di Origine Ambientale} quindi bisogna tenere conto di tutti i vincoli ambientali, normativi, temporali ed economici e chiedere consigli ad esperti del mestiere(notai, ambientalisti,investiri).
Si svolge una \textbf{Analisi della Realtà} in cui un vincolo diviene una opportunità, ossia si crea una situazione nuova e di successo.

Infine si \textbf{Definiscono gli obiettivi del progetto in termini quantitativi sintetici}:
Per definire gli obiettivi in maniera realistica e concreta, quindi senza andare in contro a perdite di tempo od uscire dai vincoli predisposti, ci sono delle tecniche come
la \textbf{\textcolor{blue}{S.M.A.R.T.}}:
\begin{itemize}
	\item \textbf{Specifici:} Deve essere dettagliato ed espresso chiaramente
	\item \textbf{Misurabili:} Quantificatore che indica la qualità (es. 10\% più grande...).
	\item \textbf{Accordati:} Deve essere concordato con tutti i membri del progetto.
	\item \textbf{Realistici:} Deve rispettare i vincoli del progetto (vedi Fig \ref{t}) e le 										capacità dei membri.
	\item \textbf{Temporalmente Definito:} Inserire una data di scandeza/consegna da rispettare.
\end{itemize}
Altra tecnica è la \textbf{\textcolor{blue}{S.W.O.T.}}, ossia Strenght, Weakness, Opportunities e Threat.
Questi punti costruscono una tabella che conterrà i vari punti di forza e debolezza sia dovuti a fattori interni(Strenght e Weakness) sia ad esterni(Opportunities e Threat):

\begin{table}[h!]
\begin{center}
\begin{tabular}{ |c|c|c|c| } 
 \hline
 S & W & O & T \\ [0.5ex] 
 \hline\hline
 cell1 & cell2 & cell3 & cell4 \\
 cell5 & cell6 & cell7 & cell8 \\
 \hline
\end{tabular}
\end{center}
\caption{Tabella S.W.O.T.}
\label{swot}
\end{table}




\subsection{Definizione di Massima del progetto:}
Consiste nel definire i requisiti del progetto che devono essere soddisfatti ed è compito del PM.
Viene divisa in:
\begin{itemize}
    \item \textbf{Definizione dei requisiti:} Sono le condizioni che il sistema deve rispettare. 		Compito del progettista elaboare una proposta di soluzione con lo scopo di
    \begin{enumerate}
        \item Identificare come deve fare il sistema informativo per rispondere alle esigenze della gente(\textbf{Requisiti funzionali}).
        \item Precisare i confini dell'applicazione e le modalità di iterazione con l'ambiente(\textbf{Requisiti di interfaccia}).
        \item Trasformare il quadro dei vincoli d’uso nei requisiti non funzionali.
        \item Definire l'elenco dei requisiti
        \item Elaborare le contradizione tra requisiti
        \item Identificare e mantenere il tracciamento tra requisiti utente e requisiti software.
    \end{enumerate}
    
    \item \textbf{Definire in linea di massima le specifiche del sistema:} architettura dei dati(database e DBMS), architettura applicativa ed interfaccia utente. 
    \item \textbf{Scelta delle modalità di realizzazione del progetto:} MakeOrBuy, riuso dei componenti esistenti, manutenzioni del sistema,formazione ed assistenza utente.
\end{itemize}


\subsection{Risk BreakDown Structure(RBS)}
Durante la stesura di un progetto il \textbf{rischio} è un evento/condizione che, nel caso dovesse succedere, avrebbe effetti negativi sul progetto stesso.
Sono presenti due fattori da tenere in considerazione: La \textbf{Probabilità} che il fenomeno avvenga e l' \textbf{Impatto} dello stesso, con le dovute conseguenze.
La gestione dei rischi è un processo di \textbf{Prevenzioni} in quanto bisogna evitare ogni possibile rischio, di \textbf{Mitigazione} ossia che bisogna adottare provvedimenti per la riduzione degli effetti indesiderati e di \textbf{Gestione delle conseguenze} predisporre in anticipo precauzioni nel caso il rischio avvenga e adottare provvedimenti per risolverlo.
Il RBS è una delle 10 aree di conoscenza del PMBOK, sono presenti 6 procedimenti che garantiscono la completa gestione dei rischi:
\begin{itemize}

 \item \textbf{Pianificazione della Gestione dei Rischi:} (Plan Risk Management)
 
 \item \textbf{Identificazione Dei Rischi}: (Identify Risk)
 
 \item \textbf{Analisi Quantitativa dei Rischi}:(Perform Qualitative Risk Analysis)
 
 \item \textbf{Analisi Quantitativa dei Rischi}:(Perform Quantitative Risk Analysis)
 
 \item \textbf{Pianificazione della Risposta ai Rischi}:(Plan Risk Responses)
 
 \item \textbf{Monitoraggio e Controllo Rischi}:(Control Risks)
\end{itemize}

Quindi il RBS serve proprio per evitare tutte le situazioni di \textbf{Crisis Managemnt}, ottimizzare i costi previsti con la dovuta gestione dei costi, aumentare le probabilità di successo e promuovere le oppurtunità.

In questa fase le attività del Project Manager sono due:
\begin{enumerate}

	\item \textbf{Individuare Fattori di Rischio}: Individuare delle aree in cui è possibile che arrivino dei rischi(punti deboli dell'azienda)
	\item \textbf{La Definizione}: Fornire una documentazione dei vari fattori di rischi, fornirne un livello quantitativo ed il motivo, come rappresentato nella tabella sottostante:
\begin{table}[h!]
\begin{center}
\begin{tabular}{ |c|c|c| } 
 	\hline
 	Fattori di rischio & Parametri & Livello \\ [0.5ex] 
 	\hline
 	Complessità gestionale & 
 	Organizzazion personale & 
 	\quantities{Ampia azienda\\Alto Rischio} \\
 				
 	\hline
 	Innovazione Tecnologica & 
 	\quantities{Utilizzo nuovo hardware,\\aggionrare apparecchiature} & 
 	\quantities{Apparecchiature recenti,\\personale esperto\\Basso rischio} \\
 				\hline
 				
\end{tabular}
\end{center}
\caption{Tabella dei Rischi}
\label{trischi}
\end{table}

Una volta stipulata la tabella di rischio si passa alla stesura della matrice di probabilità di impatto.\\
Date come variabili: \(F_r = probability'\) e \(x_0 = impact\) l'indice di probabilità \(F_r\) rappresenta la probabilità di accadimento di un evento suddiviso in:
\begin{quote}
\(P = indexProbability\) in percentuale.\\
Bassa: \begin{math}P < 20\% \Longrightarrow F_r = 1\end{math}.\\
Media: \begin{math} 20 \leq P \leq 50\% \Longrightarrow F_r = 2\end{math}.\\
Alta: \begin{math}P < 50\% \Longrightarrow F_r = 3\end{math}.\\
\end{quote}

Il nostro indice d'impatto \(x_0\) viene calcolato in base al contesto:
\begin{quote}
Tempo: \begin{math} x_t = 1	\wedge x_t = 2 \wedge x_t = 3\end{math} influenza il tempo richiesto.\\
Costo: \begin{math} x_c = 1	\wedge x_c = 2 \wedge x_c = 3\end{math} incremento dei costi.\\
Prestazioni: \begin{math} x_p = 1	\wedge x_p = 2 \wedge x_p = 3\end{math} riduzione della qualità.	\\
\end{quote}
Alla fine l'indice di impatto sarà la somma di \(x_0 = x_t + x_c + x_p\) che sarà in un range di \(3\leq x_0\leq9\).
Una volta quantificati i vari fattori di rischi, calcolata la probabilità e l'impatti si passa alla stesura della matrice dei Rischi:
La prima parte consiste nella descrizione dei rischi e del livello prima di effettuare le azioni preventive
\begin{table}[h!]
\begin{tabular}{|l|l|l|l|l|}
\hline
 &  & \multicolumn{3}{l|}{Valutazione Iniziale} \\ \cline{3-5} 
\multirow{-2}{*}{WBS} & \multirow{-2}{*}{\begin{tabular}[c]{@{}l@{}}Descrizione del Rischio\\ (Causa ed Effetto)\end{tabular}} & {\color[HTML]{000000} Probabilità} & {\color[HTML]{000000} Impatto} & {\color[HTML]{000000} Livello di Rischio} \\ \hline
 &  &  &  &  \\ \hline
 &  &  &  &  \\ \hline
 &  &  &  &  \\ \hline
\end{tabular}
\caption{Matrice Pt1}
\end{table}

La seconda parte invece avrà dei valori di alcuni rischi abbassati in quanto il PM avrà presto le precauzioni necessari da avere un livello di sicurezza sufficiente
\begin{table}[h!]
\begin{tabular}{|l|l|l|l|l|l|l|}
\hline
\multirow{2}{*}{\begin{tabular}[c]{@{}l@{}}Azioni \\ Preventive\end{tabular}} & \multicolumn{3}{l|}{Valutazione Finale} & \multicolumn{3}{l|}{Risposta al Rischio} \\ \cline{2-7} 
 & Probabilità & Impatto & \begin{tabular}[c]{@{}l@{}}Livello \\ di Rischio\end{tabular} & \begin{tabular}[c]{@{}l@{}}Condizioni \\ Di Allerta\end{tabular} & \begin{tabular}[c]{@{}l@{}}Azione \\ Correttiva\end{tabular} & Responsabile \\ \hline
 &  &  &  &  &  &  \\ \hline
 &  &  &  &  &  &  \\ \hline
 &  &  &  &  &  &  \\ \hline
\end{tabular}
\caption{Matrice pt2}
\end{table}


\end{enumerate}
\end{document}